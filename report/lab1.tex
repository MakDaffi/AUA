\documentclass[spec, och, labwork]{shiza}
% параметр - тип обучения - одно из значений:
%    spec     - специальность
%    bachelor - бакалавриат (по умолчанию)
%    master   - магистратура
% параметр - форма обучения - одно из значений:
%    och   - очное (по умолчанию)
%    zaoch - заочное
% параметр - тип работы - одно из значений:
%    referat    - реферат
%    coursework - курсовая работа (по умолчанию)
%    diploma    - дипломная работа
%    pract      - отчет по практике
% параметр - включение шрифта
%    times    - включение шрифта Times New Roman (если установлен)
%               по умолчанию выключен
\usepackage{subfigure}
\usepackage{tikz,pgfplots}
\pgfplotsset{compat=1.5}
\usepackage{float}

%\usepackage{titlesec}
\setcounter{secnumdepth}{4}
%\titleformat{\paragraph}
%{\normalfont\normalsize}{\theparagraph}{1em}{}
%\titlespacing*{\paragraph}
%{35.5pt}{3.25ex plus 1ex minus .2ex}{1.5ex plus .2ex}

\titleformat{\paragraph}[block]
{\hspace{1.25cm}\normalfont}
{\theparagraph}{1ex}{}
\titlespacing{\paragraph}
{0cm}{2ex plus 1ex minus .2ex}{.4ex plus.2ex}

% --------------------------------------------------------------------------%


\usepackage[T2A]{fontenc}
\usepackage[utf8]{inputenc}
\usepackage{graphicx}
\graphicspath{ {./images/} }
\usepackage{tempora}

\usepackage[sort,compress]{cite}
\usepackage{amsmath}
\usepackage{amssymb}
\usepackage{amsthm}
\usepackage{fancyvrb}
\usepackage{listings}
\usepackage{listingsutf8}
\usepackage{longtable}
\usepackage{array}
\usepackage[english,russian]{babel}

% \usepackage[colorlinks=true]{hyperref}
\usepackage{url}

\usepackage{underscore}
\usepackage{setspace}
\usepackage{indentfirst} 
\usepackage{mathtools}
\usepackage{amsfonts}
\usepackage{enumitem}
\usepackage{tikz}
\usepackage{minted}

\newcommand{\eqdef}{\stackrel {\rm def}{=}}
\newcommand{\specialcell}[2][c]{%
\begin{tabular}[#1]{@{}c@{}}#2\end{tabular}}

\renewcommand\theFancyVerbLine{\small\arabic{FancyVerbLine}}

\newtheorem{lem}{Лемма}

\begin{document}

% Кафедра (в родительном падеже)
\chair{}

% Тема работы
\title{Классификация бинарных отношений и системы замыканий}

% Курс
\course{3}

% Группа
\group{331}

% Факультет (в родительном падеже) (по умолчанию "факультета КНиИТ")
\department{факультета КНиИТ}

% Специальность/направление код - наименование
%\napravlenie{09.03.04 "--- Программная инженерия}
%\napravlenie{010500 "--- Математическое обеспечение и администрирование информационных систем}
%\napravlenie{230100 "--- Информатика и вычислительная техника}
%\napravlenie{231000 "--- Программная инженерия}
\napravlenie{100501 "--- Компьютерная безопасность}

% Для студентки. Для работы студента следующая команда не нужна.
% \studenttitle{Студентки}

% Фамилия, имя, отчество в родительном падеже
\author{Окунькова Сергея Викторовича}

% Заведующий кафедрой
% \chtitle{} % степень, звание
% \chname{}

%Научный руководитель (для реферата преподаватель проверяющий работу)
\satitle{аспирант} %должность, степень, звание
\saname{В. Н. Кутин}

% Руководитель практики от организации (только для практики,
% для остальных типов работ не используется)
% \patitle{к.ф.-м.н.}
% \paname{С.~В.~Миронов}

% Семестр (только для практики, для остальных
% типов работ не используется)
%\term{8}

% Наименование практики (только для практики, для остальных
% типов работ не используется)
%\practtype{преддипломная}

% Продолжительность практики (количество недель) (только для практики,
% для остальных типов работ не используется)
%\duration{4}

% Даты начала и окончания практики (только для практики, для остальных
% типов работ не используется)
%\practStart{30.04.2019}
%\practFinish{27.05.2019}

% Год выполнения отчета
\date{2022}

\maketitle

% Включение нумерации рисунков, формул и таблиц по разделам
% (по умолчанию - нумерация сквозная)
% (допускается оба вида нумерации)
% \secNumbering

%-------------------------------------------------------------------------------------------
\tableofcontents

\section{Постановка задачи}

Цель работы:

Изучение основных свойств бинарных отношений и операций замыкания бинарных отношений.

Порядок выполнения работы:
    \begin{enumerate}
        \item Разобрать основные определения видов бинарных отношений и разработать
        алгоритмы классификации бинарных отношений.
        \item Изучить свойства бинарных отношений и рассмотреть основные системы
        замыкания на множестве бинарных отношений.
        \item Разработать алгоритмы построения основных замыканий бинарных отношений.
    \end{enumerate}

\section{Теоретические сведения по рассмотренным темам с их обоснованием}

\subsection{Определение бинарного отношения:}
        
        Подмножества декартова произведения $A \cdot B$ множеств $A$ и $B$ называются \textbf{бинарными отношениями} между элементами множеств $A, B$ и обозначаются строчными греческими буквами: $\rho, \sigma, \rho_1, \rho_2, ...$. 

        Для бинарного отношения $\rho \subset A \cdot B$ область определения $D_\rho$ и множество значений $E_\rho$ определяется как подмножества соответствующих множеств $A$ и $B$ по следующим формулам:

        \[D_\rho = \left\{ a: (a, b) \in \rho \text{ для некоторого } b \in B \right\}\]
        \[E_\rho = \left\{ b: (a, b) \in \rho \text{ для некоторого } a \in A \right\}\]

\subsection{Свойства бинарных отношений:}

        Бинарное отношени является:

        \begin{enumerate}
            \item \textit{рефлексивным}, если $(a, a) \in \rho$ для любого $a \in A$;
            \item \textit{антирефлексивным}, если $(a, a) \notin \rho$ для любого $a \in A$;
            \item \textit{симметричным}, если $(a, b) \in \rho \Rightarrow (b, a) \in \rho$;
            \item \textit{антисимметричным}, если $(a, b) \in \rho \text{ и } (b, a) \in \rho \Rightarrow a = b$;
            \item \textit{транзитивным}, если $(a, b) \in \rho \text{ и } (b, c) \in \rho \Rightarrow (a, c) \in \rho$
            \item \textit{антитранзитивным}, если $(a, b) \in \rho \text{ и } (b, c) \in \rho \Rightarrow (a, c) \notin \rho$
        \end{enumerate}

\subsection{Классификация бинарных отношений:}

        Классификация бинарных отношений напрямую определяются их свойствами.

        \begin{enumerate}
            \item Рефлексивное транзитивное отношение называется отношением квазипорядка.
            \item Рефлексивное симметричное транзитивное отношение называется отношением эквивалентности.
            \item Рефлексивное антисимметричное транзитивное отношение называется отношением (частичного) порядка.
            \item Антирефлексивное, антисимметричное и транзитивное отношение называется отношением строгого порядка.
        \end{enumerate}

\subsection{Замыкание отношения:}
        
        \textbf{Замыканием отношения} $R$ относительно свойства $P$ называется такое множество $R^*$, что:
        
        \begin{enumerate}
            \item $R \subset R^*$.
            \item $R^*$ обладает свойством $P$.
            \item $R^*$ является подмножеством любого другого отношения, содержащего $R$ и обладающего свойством $P$.
        \end{enumerate}

\section{Результаты работы}

        \subsection{Описание алгоритма классификации бинарных отношений}
            \begin{enumerate}
                \item Проверка на бинарного отношения на рефлексивность:

                Вход: матрица бинарного отношения

                Выход: "Множество рефлексивно" или "Множество антирфелексивно" или "Множество не рефлексивно"
                
                На вход подается матрица бинарного отоношения A. 
                
                Шаг1. Суммирование элементов на главной диагонали ($sum = \sum\limits_{i=1}^n a[i][i]$) с помощью цикла. 
                
                Шаг2. Если $sum = n$, то отношение является рефлексивным, иначе, если $sum = 0$, то отношение является 
                антирефлексивным, иначе отношение является не рефлексивным.

                \item Проверка на бинарного отношения на симетричность:
                
                Вход: матрица бинарного отношения

                Выход: "Множество симетрично" или "Множество антисиметрично" или "Множество не симетрично"

                На вход подается матрица бинарного отоношения A. 

                Шаг1. Транспонируем A, чтобы получить $A^T$.

                Шаг2. Если $A = A^T$, то бинарное отношение будет является симетричным, иначе нужно проверить матрицу на 
                антисимитричность. 

                Шаг3. Получим матрицу $B = A * A^T$.

                Шаг4. Проходимся дву вложенными циклами по матрице B. Если для любого $i \neq j b[i][j] = 0$, где $b[i][j]$ элемент 
                матрицы B, то отношение является антисиметричным, иначе отношение не симетрично.

                \item Проверка бинарного отношения на транзитивность:
                
                Вход: матрица бинарного отношения

                Выход: "Множество транзитивно" или "Множество антитранзитивно" или "Множество не транзитивно"
                
                На вход подается матрица бинарного отношения A. 
                
                Шаг1. Возвести A в квадрат. 
                
                Шаг2. Сравнить полученную и исходную матрицу.
                
                Шаг3. Если $A^2 \leq A$, то бинарное отношение транзитивно, иначе запускается проверка на антитранзитивность
                
                Шаг4. Запускается три вложенных цикла по элементам матрицы A, если на каком-то шаге
                $a[i][k]=1$ и $a[k][j]=1$ и $a[i][j]=1$, то отношение не будет являтся транзитивным, иначе отношение будет
                являтся антитранзитивным.

            \end{enumerate}
        
        \subsection{Описание алгоритмов построения основных замыканий бинарных отношений}
            \begin{enumerate}
                
                \item Замыкание бинарного отношения относительно рефлексивности:
                
                Вход: матрица бинарного отношения

                Выход: исходная матрица бинарного отношения, замкнутая относительно рефлексивности

                На вход подается матрица бинарного отношения A.

                Шаг1. Запуск цикла по i от 1 до n, в котором каждому элементу $a[i][i]$ присвоить еденицу.

                \item Замыкание бинарного отношения относительно симетричности:
                
                Вход: матрица бинарного отношения

                Выход: исходная матрица бинарного отношения, замкнутая относительно симитричности

                На вход подается матрица бинарного отношения A.

                Шаг1. Запуск двух вложенных циклов по всем элементам матрицы A, в которых каждому $a[i][j]$ присваивается значение элемент
                $a[j][i]$.

                \item Замыкание бинарного отношения относительно транзитивности:
                
                Вход: матрица бинарного отношения

                Выход: исходная матрица бинарного отношения, замкнутая относительно транзитивности

                На вход подается матрица бинарного отношения A.

                Шаг1. Запуск трех вложенных цикла по всем элементам матрицы A, в которых установки значения $a[i][j] = 1$, если
                $a[i][k] = 1$ и $a[k][j] = 1$.
            \end{enumerate}
    
        \subsection{Коды программ, реализующей рассмотренные алгоритмы}

            \inputminted[fontsize=\small]{python}{../code/lab1.py}
    
        \subsection{Результаты тестирования программ}

        \begin{figure}[H]
            \centering      %размер рисунка       здесь находится название файла рисунка, без указания формата
            \includegraphics[width=1.\textwidth]{1}
            \caption{Тест 1}
            \label{fig:image1}
        \end{figure}
        
        \begin{figure}[H]
            \centering      %размер рисунка       здесь находится название файла рисунка, без указания формата
            \includegraphics[width=1.\textwidth]{2}
            \caption{Тест 2}
            \label{fig:image1}
        \end{figure}

        \begin{figure}[H]
            \centering      %размер рисунка       здесь находится название файла рисунка, без указания формата
            \includegraphics[width=1.\textwidth]{3}
            \caption{Тест 3}
            \label{fig:image1}
        \end{figure}

        \begin{figure}[H]
            \centering      %размер рисунка       здесь находится название файла рисунка, без указания формата
            \includegraphics[width=1.\textwidth]{4}
            \caption{Тест 4}
            \label{fig:image1}
        \end{figure}

        \subsection{Оценки сложности рассмотренных алгоритмов}

        \subsubsection{Алгоритм определения рефлексивности}

            Сложность выполнения проверки на рефлексивность или антирефлексивность определяется как $O(n)$.

        \subsubsection{Алгоритм определения симметричности}

            Cложность транспонирования в numpy определяется как $O(n^{3/2}log \text{ } n)$, сложность умножения матриц 
            определяется как $O(n^3)$, сложность сравнение двух матриц поэлементно определяется как $O(n^2)$. 
            Отсюда можно сделать вывод, что в случае, если наше отношение будет симметричным,
            что будет являтся лучшим случаем раоты алгоритма,то общая сложность алгоритма будет определятся как 
            $O(n^{3/2}log \text{ } n + n^2) = O(n^{3/2}log \text{ } n)$,
            иначе, в худшем случае, сложность будет определятся как $O(n^{3/2}log \text{ } n + n^2 + n^2 + n^3) = O(n^3)$

        \subsubsection{Алгоритм определения транзитивности}

            Из всего выше сказанного очевидно, что сложность проверки на транзитивность или антитранзитивность составляет $O(n^3)$,
            так как в нем используется умножение, сравнение матриц и тройной цикл для проверки рефлексивности в худшем случае матриц.

        \subsubsection{Алгоритм классификации}
            Сложность выполнения самого алгоритма классификации бинарных отношений реализованно через питоновский словарь 
            и оператор if, поэтому является константной ($O(1)$), если не учитывать сложность выполнения проверки свойств отношения.

        \subsubsection{Построение замыкания рефлексивности}
        
            Так как весь алгоритм строится на заполнении главной диагонали матрицы 1, то его сложность состовляет $O(n)$.

        \subsubsection{Построение замыкания симметричности}

            Для посторения замыкания симметричности используются вложенный цикла, поэтому сложность алгоритма
            определяется как $O(n^2)$.

        \subsubsection{Построение замыкания транзитивности}

            Для посторения замыкания транзитивности используются два вложенный цикла, поэтому сложность алгоритма
            определяется как $O(n^3)$.
    
\conclusion

В рамках данной лабораторной работы были рассмотренны теоритические основы свойств бинарных отношений, их видов и методов
их замыкания по каждому из свойств. На основе этой теоретической части была смоделирована программа, которая способна
определить свойства заданного множества, его вид и построить систему замыкания по каждому из основных свойств бинарного
отношения.

\end{document}
